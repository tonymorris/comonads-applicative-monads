\begin{frame}[fragile]
\frametitle{Familiarity}
\framesubtitle{Identifying the pattern}
\begin{block}{Turning a list of potentially \lstinline{null} into a potentially \lstinline{null} list}
\note{sequence :: [Maybe a] -> Maybe [a]}
\begin{lstlisting}[style=language]
args(list)
  result = new List;
  foreach el in list
    if(el == null)
      return null;
    else
      result.add(el);
  return result;
\end{lstlisting}
\end{block}
\end{frame}

\begin{frame}[fragile]
\frametitle{Familiarity}
\framesubtitle{Identifying the pattern}
\begin{block}{Applying a list of functions to a single value}
\note{sequence :: [t -> a] -> t -> a}
\begin{lstlisting}[style=language]
args(list, t)
  result = new List;
  foreach el in list
    result.add(el(t));
  return result;
\end{lstlisting}
\end{block}
\end{frame}

\begin{frame}[fragile]
\frametitle{Familiarity}
\framesubtitle{Identifying the pattern}
\begin{block}{These expressions share structure}
\begin{lstlisting}[style=language]
List (MaybeNull a) -> MaybeNull (List a)
List ((t ->)    a) -> (t ->)    (List a)
List (m         a) -> m         (List a)
\end{lstlisting}
\end{block}
Commonly called \lstinline{sequence}.
\end{frame}

\begin{frame}[fragile]
\frametitle{Familiarity}
\framesubtitle{Identifying the pattern again}
\begin{block}{Keep elements of a list matching a predicate with potential \lstinline{null}}
\note{filterM :: (a -> Maybe Bool) -> [a] -> Maybe [a]}
\begin{lstlisting}[style=language]
args(pred, list)
  result = new List;
  foreach el in list
    ans = pred(el);
    if(ans == null)
      return null;
    else if(ans)
      result.add(el);
  return result;
\end{lstlisting}
\end{block}
\end{frame}

\begin{frame}[fragile]
\frametitle{Familiarity}
\framesubtitle{Identifying the pattern again}
\begin{block}{Keep elements of a list matching a predicate with argument passing}
\note{filterM :: (a -> t -> Bool) -> [a] -> t -> [a]}
\begin{lstlisting}[style=language]
args(pred, list, t)
  result = new List;
  foreach el in list
    if(pred(el, t))
      result.add(el);
  return result;
\end{lstlisting}
\end{block}
\end{frame}

\begin{frame}[fragile]
\frametitle{Familiarity}
\framesubtitle{Identifying the pattern again}
\begin{block}{These expressions share structure}
\begin{lstlisting}[style=language]
(a -> MaybeNull Bool) -> List a -> MaybeNull (List a)
(a -> (t ->)    Bool) -> List a -> (t ->)    (List a)
(a -> m         Bool) -> List a -> m         (List a)
\end{lstlisting}
\end{block}
Commonly called \lstinline{filter}.
\end{frame}

\begin{frame}[fragile]
\frametitle{Familiarity}
\framesubtitle{Identifying the pattern, once again}
\begin{block}{Find the first element matching a predicate with potential \lstinline{null}}
\note{findM :: (a -> Maybe Bool) -> [a] -> Maybe a}
\begin{lstlisting}[style=language]
args(pred, list)
  result = new List;
  foreach el in list
    ans = pred(el);
    if(ans == null)
      return null;
    else if(ans)
      return a;
  return null;
\end{lstlisting}
\end{block}
\end{frame}

\begin{frame}[fragile]
\frametitle{Familiarity}
\framesubtitle{Identifying the pattern, once again}
\begin{block}{Find the first element matching a predicate with argument passing}
\note{findM :: (a -> t -> Bool) -> [a] -> t -> a}
\begin{lstlisting}[style=language]
args(pred, list, t)
  foreach el in list
    ans = pred(el, t);
    if(ans)
      return true;    
  return false;
\end{lstlisting}
\end{block}
\end{frame}

\begin{frame}[fragile]
\frametitle{Familiarity}
\framesubtitle{Identifying the pattern, once again}
\begin{block}{These expressions share structure}
\begin{lstlisting}[style=language]
(a -> MaybeNull Bool) -> List a -> MaybeNull Bool
(a -> (t ->)    Bool) -> List a -> (t ->)    Bool
(a -> m         Bool) -> List a -> m         Bool
\end{lstlisting}
\end{block}
Commonly called \lstinline{find}.
\end{frame}

\begin{frame}[fragile]
\frametitle{Familiarity}
\framesubtitle{Identifying the pattern, last time}
\begin{block}{Turn a list of lists into a list}
\note{join :: [[a]] -> [a]}
\begin{lstlisting}[style=language]
args(list)
  result = new List;
  foreach el in list
    result.append(el);
  return result;
\end{lstlisting}
\end{block}
\end{frame}

\begin{frame}[fragile]
\frametitle{Familiarity}
\framesubtitle{Identifying the pattern, last time}
\begin{block}{Turn a potential \lstinline{null} of potential \lstinline{null} into a potential \lstinline{null}}
\note{join :: Maybe (Maybe a) -> Maybe a}
\begin{lstlisting}[style=language]
args(value)
  if(value == null)
    return null;
  else
    return value.get;
\end{lstlisting}
\end{block}
\end{frame}

\begin{frame}[fragile]
\frametitle{Familiarity}
\framesubtitle{Identifying the pattern, last time}
\begin{block}{Apply to the argument, then apply to the argument}
\note{join :: (t -> t -> a) -> t -> a}
\begin{lstlisting}[style=language]
args(f, t)
  return f(t, t);
\end{lstlisting}
\end{block}
\end{frame}

\begin{frame}[fragile]
\frametitle{Familiarity}
\framesubtitle{Identifying the pattern}
\begin{block}{These expressions share structure}
\begin{lstlisting}[style=language]
List      (List a)      -> List      a
MaybeNull (MaybeNull a) -> MaybeNull a
(t ->)    ((t ->)    a) -> (t ->)    a
m         (m         a) -> m         a
\end{lstlisting}
\end{block}
Commonly called \lstinline{join}.
\end{frame}
